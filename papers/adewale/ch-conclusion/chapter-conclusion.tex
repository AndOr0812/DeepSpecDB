\section{Conclusion}
In this Thesis Paper, we have described at a high-level the implementation of our main-memory serial key-value store. We gave a quick overview of main-memory databases, B+-Trees and Masstree in section \ref{sec:Background}. We discuss out key-value store implementation, B+-Tree with cursor implementation and why B+-Tree sequential operations take amortized constant time in section \ref{sec:Our System}. We also learn more about SQLite's \cite{SQLite} B+-Tree with cursor scheme and how it is used in our system. We evaluated the behaviour of our B+-Tree library and KVStore library in section \ref{sec:Evaluation}. Here we saw that the B+-Tree with cursor idea does indeed yield amortized constant time sequential\textbf{get} and \textbf{put} operations (traditional B+-Tree's can only guarantee constant time \textbf{get} operations). Moreover, we have shown that partially sorting a sequence of inputs leads to performance gains for our B+-Tree and KVStore. We have also shown that Masstree's trie of B+-Tree scheme can properly handle long common prefixes. Finally, we briefly presented a way to extend the B+-Tree cursor to a KVStore Cursor to further speed up sequential and partially sorted queries in our system. 
